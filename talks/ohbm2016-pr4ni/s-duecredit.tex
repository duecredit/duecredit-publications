%
%
% COPYRIGHT: Yaroslav Halchenko 2016
%
% LICENSE: MIT
%
%  Permission is hereby granted, free of charge, to any person obtaining a copy
%  of this software and associated documentation files (the "Software"), to deal
%  in the Software without restriction, including without limitation the rights
%  to use, copy, modify, merge, publish, distribute, sublicense, and/or sell
%  copies of the Software, and to permit persons to whom the Software is
%  furnished to do so, subject to the following conditions:
%
%  The above copyright notice and this permission notice shall be included in
%  all copies or substantial portions of the Software.
%
%  THE SOFTWARE IS PROVIDED "AS IS", WITHOUT WARRANTY OF ANY KIND, EXPRESS OR
%  IMPLIED, INCLUDING BUT NOT LIMITED TO THE WARRANTIES OF MERCHANTABILITY,
%  FITNESS FOR A PARTICULAR PURPOSE AND NONINFRINGEMENT. IN NO EVENT SHALL THE
%  AUTHORS OR COPYRIGHT HOLDERS BE LIABLE FOR ANY CLAIM, DAMAGES OR OTHER
%  LIABILITY, WHETHER IN AN ACTION OF CONTRACT, TORT OR OTHERWISE, ARISING FROM,
%  OUT OF OR IN CONNECTION WITH THE SOFTWARE OR THE USE OR OTHER DEALINGS IN
%  THE SOFTWARE.
%

\subsection{DueCredit}

\newcommand{\duecredititem}[1]{\begin{flushright}\textcolor{red}{\emph{#1}}\end{flushright}}

\begin{frame}{Why?} %  i.e. problems with (neuroimaging) data management}
% TODO: make alert
% TODO: flush clarifications to the right?
%  \begin{block}
    \begin{itemize}
    \item Our (methods, software, datasets authors) \textbf{work is
        not cited adequately}
    \item Even if cited, \textbf{version information is often omitted}
    \item Absent visibility of contributions to existing projects
      fosters \textbf{prima ballerina projects}
    \item It is \textbf{tedious to collect/format references} for
      publications using our products
    \end{itemize}
% \end{block}
\end{frame}

\begin{frame}{DueCredit's approach}
%\avatar{lab/matteo.jpg}

Make it \textbf{VERY EASY} to
\begin{itemize}
\item \textbf{collect references} for methods, software, and data
  \textbf{actually used} by the analysis
\item \textbf{accumulate reference} information through the entirety of the
  research project
\item \textbf{collect references} for methods implemented in the
  software and its users
\item \textbf{request references} to cite using desired format: LaTeX
  + BibTeX, rendered citations in variety of styles, etc
\end{itemize}
\end{frame}

\begin{frame}[t,fragile]{Example: \texttt{examples/example\_scipy.py}}

\begin{minted}[fontsize=\scriptsize]{python}
# A tiny analysis script to demostrate duecredit
#
# Import of duecredit is not necessary if you just run this script with
# python -m duecredit
# import duecredit  # Just to enable duecredit
from scipy.cluster.hierarchy import linkage
from scipy.spatial.distance import pdist
from sklearn.datasets import make_blobs

print("I: Simulating 4 blobs")
data, true_label = make_blobs(centers=4)

dist = pdist(data, metric='euclidean')

Z = linkage(dist, method='single')
print("I: Done clustering 4 blobs")
\end{minted}
\end{frame}


\begin{frame}[t,fragile]{What is it for?}
\begin{Verbatim}[commandchars=\\\{\},fontsize=\scriptsize]
> python -m duecredit examples/example_scipy.py
I: Simulating 4 blobs
I: Done clustering 4 blobs

DueCredit Report:
- scipy (v 0.14.1) [1]
  - scipy.cluster.hierarchy:linkage (v 0.14.1) [2]
- sklearn (v 0.16.1) [3]

2 packages cited
0 modules cited
1 functions cited

References
----------

[1] Jones, E. et al., 2001. SciPy: Open source scientific tools for ...     Python.
[2] Sibson, R., 1973. SLINK: an optimally efficient algorithm for ...       the single-link cluster method. The Computer Journal, 16(1), pp.30–34.
[3] Pedregosa, F. et al., 2011. Scikit-learn: Machine learning in Python.   The Journal of Machine Learning Research, 12, pp.2825–2830.
\end{Verbatim}
\end{frame}


\begin{frame}[t,fragile]{A bit bigger example}
\begin{Verbatim}[commandchars=\\\{\},fontsize=\scriptsize]
> python -m duecredit /usr/bin/nosetests mvpa2.tests.test_transerror
........
DueCredit Report:
- libsvm (v None) [1]
- mvpa2 (v 2.4) [2]
  - mvpa2.clfs.SVM (v 2.4) [3]
  - mvpa2.clfs.smlr:SMLR (v 2.4) [4]
  - mvpa2.clfs.transerror:_call (v 2.4) [5]
  - mvpa2.featsel.rfe:_train (v 2.4) [6]
  - mvpa2.measures.searchlight:_call (v 2.4) [7]
- scipy (v 0.14.1) [8]
- sklearn (v 0.16.1) [9]

4 packages cited
1 modules cited
4 functions cited

References
----------

[1] Chang, C.-C. & Lin, C.-J., 2011. LIBSVM. ACM Trans. Intell. Syst. ...     Technol., 2(3), pp.1–27.
[2] Hanke, M. et al., 2009. PyMVPA: a Python Toolbox for Multivariate ...     Pattern Analysis of fMRI Data. Neuroinform, 7(1), pp.37–53.
[3] Vapnik, V., 1995. The Nature of Statistical Learning Theory, New ...      York: Springer.
...
\end{Verbatim}
\end{frame}

\begin{frame}[t,fragile]{A bit bigger example: alternative output}
\begin{Verbatim}[fontsize=\scriptsize]
> duecredit summary --format=bibtex
@article{Hanke_2009, title={PyMVPA: a unifying approach to the analysis of neuroscientific data}, volume={3}, ISSN={1662-5196}, url={http://dx.doi.org/10.3389/neuro.11.003.2009}, DOI={10.3389/neuro.11.003.2009}, journal={Frontiers in Neuroinformatics}, publisher={Frontiers Media SA}, author={Hanke, Michael}, year={2009}}
@article{pedregosa2011scikit,
        title={Scikit-learn: Machine learning in Python},
        author={Pedregosa, Fabian and Varoquaux, Ga{"e}l and Gramfort,
        Alexandre and Michel, Vincent and Thirion, Bertrand and Grisel,
        Olivier and Blondel, Mathieu and Prettenhofer, Peter and Weiss,
        Ron and Dubourg, Vincent and others},
        journal={The Journal of Machine Learning Research},
        volume={12},
        pages={2825--2830},
        year={2011},
        publisher={JMLR. org}
        }
@article{Kriegeskorte_2006, title=...
...
\end{Verbatim}
\end{frame}


\begin{frame}[fragile]{Contribute HOWTO 101.1: Use in your software}

\begin{enumerate}
\item copy \texttt{duecredit/stub.py} to your codebase, e.g.
\begin{Verbatim}[fontsize=\scriptsize]
wget -q -O /path/tomodule/yourmodule/due.py \
  https://raw.githubusercontent.com/
     duecredit/duecredit/master/duecredit/stub.py
\end{Verbatim}

\item Then use \texttt{duecredit} import due and necessary entries in your code as

\begin{minted}[fontsize=\scriptsize]{python}
from .due import due, Doi
\end{minted}

to provide reference for the entire module just use e.g.

\begin{minted}[fontsize=\scriptsize]{python}
due.cite(Doi("1.2.3/x.y.z"), description="Solves all your problems",
         path="magicpy")
\end{minted}

To provide a reference for a function or a method, use dcite decorator

\begin{minted}[fontsize=\scriptsize]{python}
@due.dcite(Doi("1.2.3/x.y.z"), description="Resolves constipation ...     issue")
def pushit():
    ...
\end{minted}
\end{enumerate}
\end{frame}

\begin{frame}[fragile]{Contribute HOWTO 101.2: Inject for other modules}
Example: \texttt{duecredit/injections/mod\_scipy.py}

\begin{minted}[fontsize=\scriptsize]{python}
from ..entries import Doi, BibTeX, Url
def inject(injector):
    injector.add('scipy', None, BibTeX("""
                    @Misc{JOP+01,
                    ...
                    }"""),
                 description="Scientific tools library",
                 tags=['implementation'])
    ...
    injector.add('scipy.cluster.hierarchy', 'linkage', BibTeX("""
                    @article{ward1963hierarchical,
                    ...
                    }"""),
                 conditions={(1, 'method'): {'ward'}},
                 description="Ward hierarchical clustering",
                 min_version='0.4.3',
                 tags=['reference'])
    ...
\end{minted}

\end{frame}

\begin{frame}
  {Get involved!}
  \begin{itemize}
    \item Use in your software and to collect references for your analysis scripts
    \item Report bugs, send pull requests/patches
    \item Provide support for other languages/environments
      \begin{description}
      \item[Matlab/Octave]
        \url{https://github.com/duecredit/duecredit/issues/20}
      \item[Java, R, C/C++, ...] \textbf{You?}
      \end{description}
    \item Spread the word
    \end{itemize}
    \begin{center}
    \vspace{4mm}
    \alert{\Huge WE NEED HELP!}
  \end{center}
  \refnote{\url{http://duecredit.org}}
\end{frame}


%%% Local Variables: 
%%% mode: latex
%%% TeX-master: "s"
%%% End: 
